\documentclass[11pt, a4paper]{article}
\usepackage[czech]{babel}
\usepackage[utf8]{inputenc}
\usepackage[IL2]{fontenc}
\usepackage[left=1.5cm,top=2.5cm,text={18cm, 25cm}]{geometry}
\usepackage{amsthm}
\usepackage{amsfonts}
\usepackage{amsmath}
\usepackage{times}

\theoremstyle{definition}
\newtheorem{definice}{Definice}
\newtheorem{veta}{Věta}

\hyphenation{jestliže}

\begin{document}

\begin{titlepage}
\begin{center}
{\Huge
\textsc{Fakulta informačních technologií\\[0.4em]
Vysoké učení technické v~Brně}}\\
{\LARGE
\vspace{\stretch{0.382}}
Typografie a publikování\,--\,2. projekt\\[0.3em]
Sazba dokumentů a matematických výrazů}\\
\vspace{\stretch{0.618}}
\end{center}
{\Large 2018 \hfill
Dominik Večeřa (xvecer23)}
\end{titlepage}

\begin{twocolumn}

\section*{Úvod}
V~této úloze si vyzkoušíme sazbu titulní strany, matematických vzorců, prostředí a dalších textových struktur obvyklých pro technicky zaměřené texty (například rovnice~(\ref{Eqn1}) nebo Definice \ref{Def1} na straně \pageref{Def1}). Rovněž si vyzkoušíme používání odkazů \verb|\ref| a \verb|\pageref|.

Na titulní straně je využito sázení nadpisu podle optického středu s~využitím zlatého řezu. Tento postup byl
probírán na přednášce. Dále je použito odřádkování se zadanou relativní velikostí 0.4em a 0.3em.
\section{Matematický text}
Nejprve se podíváme na sázení matematických symbolů a~výrazů v~plynulém textu včetně sazby definic a vět s~využitím balíku \verb|amsthm|. Rovněž použijeme poznámku pod čarou s~použitím příkazu \verb|\footnote|. Někdy je vhodné
použít konstrukci \verb|${}$|, která říká, že matematický text nemá být zalomen.

\begin{definice} \label{Def1}
Turingův stroj \textit{(TS) je definován jako šestice tvaru $M = (Q, \Sigma, \Gamma, \delta, q_0, q_F)$, kde:}
\end{definice}

\begin{itemize}
    \item \textit{Q je konečná množina} vnitřních (řídicích) stavů,
    \item \textit{$\Sigma$ je konečná množina symbolů nazývaná} vstupní abeceda, $\Delta \notin \Sigma$,
    \item \textit{$\Gamma$ je konečná množina symbolů, $\Sigma \subset \Gamma, \Delta \subset \Gamma$, nazývaná} pásková abeceda,
    \item \textit{$\delta:(Q$\textbackslash$\{q_F\})\!\times\!\Gamma\!\rightarrow Q\!\times\!(\Gamma\!\cup\!\{\!L,R\}\!)$, kde L, R $\notin\Gamma$, je parciální} přechodová funkce,
    \item \textit{$q_0$ je} počáteční stav, $q_0 \in Q$ \textit{a}
    \item \textit{$q_F$ je} koncový stav, $q_F \in Q$.
\end{itemize}

Symbol $\Delta$ značí tzv. \textit{blank} (prázdný symbol), který se vyskytuje na místech pásky, která nebyla ještě použita (může ale být na pásku zapsán i později).

\textit{Konfigurace pásky} se skládá z~nekonečného řetězce, který reprezentuje obsah pásky a pozice hlavy na tomto řetězci. Jedná se o~prvek množiny \{ $ \gamma \Delta^\omega$ \textbar $\gamma \in \Gamma^* \} \times \mathbb{N} $.\footnote{Pro libovolnou abecedu $\Sigma$ je $\Sigma^\omega$ množina všech \textit{nekonečných} řetězců nad $\Sigma$, tj. nekonečných posloupností symbolů ze $\Sigma$. Pro připomenutí: $\Sigma^*$ je množina všech \textit{konečných} řetězců nad $\Sigma$.} \textit{Konfiguraci pásky} obvykle zapisujeme jako $\Delta xyz \underline{z}x \Delta \dots$ (podtržení značí pozici hlavy). \textit{Konfigurace stroje} je pak dána stavem řízení a konfigurací pásky. Formálně se jedná o~prvek množiny $Q \times \{ \gamma \Delta^\omega$ \textbar $ \gamma \in \Gamma^*\} \times \mathbb{N}.$

\subsection{Podsekce obsahující větu a odkaz}
\begin{definice} \label{Def2}
Řetězec $w$ nad abecedou $\Sigma$ je přijat TS \textit{M jestliže M při aktivaci z~počáteční konfigurace pásky\linebreak $\underline{\Delta}w\Delta\dots$ a~počátečního stavu $q_0$ zastaví přechodem do koncového stavu $q_F$, tj. ($q_0, \Delta w\Delta^\omega, 0) \underset{M}{\overset{*}{\vdash}} (q_F, \gamma, n)$ pro nějaké $\gamma \in~\Gamma^*$ a $n \in \mathbb{N}$}.

\textit{Množinu L(M) = \{$w \mid w$ je přijat TS M\} $\subseteq \Sigma^*$ nazýváme} jazyk přijímaný TS \textit{M}.
\end{definice}

Nyní si vyzkoušíme sazbu vět a důkazů opět s~použitím balíku \verb|amsthm|.

\begin{veta}
\textit{Třída jazyků, které jsou přijímány TS, odpovídá} rekurzivně vyčíslitelným jazykům.
\end{veta}

\begin{proof}
V~důkaze vyjdeme z~Definice \ref{Def1} a \ref{Def2}.
\end{proof}

\section{Rovnice a odkazy}
Složitější matematické formulace sázíme mimo plynulý
text. Lze umístit několik výrazů na jeden řádek, ale pak je
třeba tyto vhodně oddělit, například příkazem \verb|\quad|.

\smallskip
\begin{center}
    $\sqrt[i]{x_i^3}$ kde $x_i$ je $i$-té sudé číslo \quad $y_i^{2\cdot y_i} \neq y_i^{y_i^{y_i}}$
\end{center}

V~rovnici (\ref{Eqn1}) jsou využity tři typy závorek s~různou explicitně definovanou velikostí.

\begin{align}
x &= \bigg\{ \Big( \big[ a + b \big] * c \Big)^d \oplus 1 \bigg\} \label{Eqn1} \\
y &= \lim_{x\to\infty} \frac{\sin^2 x + \cos^2 x}{\frac{1}{\log_{10}x}} \nonumber
\end{align}

V~této větě vidíme, jak vypadá implicitní vysázení limity $\lim_{n \to \infty} f(n)$ v~normálním odstavci textu. Podobně je to i s~dalšími symboly jako $\sum_{i=1}^n 2^i$ či $\bigcup_{A \in \mathcal{B}} A$. V~případě vzorců $\lim\limits_{n\to\infty} f(n)$ a $\sum\limits_{i=1}^n 2^i$ jsme si vynutili méně úspornou sazbu příkazem \verb|\limits|.

\begin{eqnarray}
\int\limits_{a}^{b} f(x) \mathrm{d}x &= & - \int_{a}^{b} g(x) \mathrm{d}x\\
\overline{\overline{A \vee B}} &\Leftrightarrow & \overline{\overline{A} \wedge \overline{B}} 
\end{eqnarray}

\section{Matice}
Pro sázení matic se velmi často používá prostředí \verb|array| a závorky (\verb|\left|, \verb|\right|).

$$ \left(
\begin{array}{ccc}
    a + b & \widehat{\xi + \omega} & \hat{\pi}\\
    \overset{\rightarrow}{a} & \overleftrightarrow{AC} & \beta 
\end{array}
\right)
= 1 \Longleftrightarrow \mathbb{Q} = \mathbb{R}$$

$$
\textbf{A} = 
\left| \left|
\begin{array}{cccc}
a_{11} & a_{12} & \dots & a_{1n} \\
a_{21} & a_{22} & \dots & a_{2n} \\
\vdots & \vdots & \ddots & \vdots \\
a_{m1} & a_{m2} & \dots & a_{mn} \\
\end{array}
\right| \right|
=
\left|
\begin{array}{cc}
    t & u~\\
    v~& w
\end{array}
\right|
= tw - uv
$$

Prostředí \verb|array| lze úspěšně využít i jinde.

$$
\binom{n}{k}
=
\Bigg\{
\begin{array}{ll}
    \frac{n!}{k!(n-k)!} & \text{pro } 0 \leq k~\leq n\ \\
    0 & \text{pro } k~< 0 \text{ nebo } k~> n \\
\end{array}
$$

\section{Závěrem}

V~případě, že budete potřebovat vyjádřit matematickou konstrukci nebo symbol a nebude se Vám dařit jej nalézt v~samotném \LaTeX u, doporučuji prostudovat možnosti balíku maker \AmS-\LaTeX.

\end{twocolumn}
\end{document}
